\documentclass{article}
\usepackage{pgfplots}
\pgfplotsset{compat=1.17}

\begin{document}
	
	\begin{figure}
		\centering
		\begin{tikzpicture}
			\begin{axis}[
				xlabel=Tempo,
				ylabel=Posição,
				title=Gráfico do Oscilador Harmônico,
				grid=both,
				width=0.8\textwidth,
				height=0.6\textwidth
				]
				\addplot[smooth, very thick, red] table {oscilador.dat}; % Aqui ele importa os dados diretamente do arquivo oscilador.dat
			\end{axis}
		\end{tikzpicture}
		\caption{Gráfico do Oscilador Harmônico.}
	\end{figure}
	
	\begin{figure}
	\centering
	\begin{tikzpicture}
		\begin{axis}[
			xlabel=Tempo,
			ylabel=Posição,
			title=Gráfico do Oscilador Harmônico para Diferentes Frequências Angulares,
			grid=both,
			width=0.8\textwidth,
			height=0.6\textwidth,
			legend pos=north west, % Posição da legenda
			]
			\addlegendentry{$\omega$ 1.0}
			\addplot [smooth, very thick, blue] table {dados_freq_1.dat};
			
			\addlegendentry{$\omega$ 2.0}
			\addplot [smooth, very thick, red, dashed] table {dados_freq_2.dat};
			
			\addlegendentry{$\omega$ 3.0}
			\addplot [smooth, very thick, green, dotted] table {dados_freq_3.dat};
			
			\addlegendentry{$\omega$ 4.0}
			\addplot [smooth, very thick, orange, dashdotted] table {dados_freq_4.dat};
			
			\addlegendentry{$\omega$ 5.0}
			\addplot [smooth, very thick, purple, densely dotted] table {dados_freq_5.dat};
		\end{axis}
	\end{tikzpicture}
	\caption{Gráfico do Oscilador Harmônico para Diferentes Frequências Angulares.}
\end{figure}
	
\end{document}
