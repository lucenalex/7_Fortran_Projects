\title{Introdução aos computadores e à linguagem Fortran}
\begin{frame}
\maketitle
\end{frame}
\begin{frame}
\frametitle{Objetivos}
\begin{itemize}
  \item Conhecer os componentes básicos de um computador;
  \item Compreender o que são números binários, octa- e hexádecimal;
  \item Aprender sobre a história da linguagem fortran.
\end{itemize}
\nocite{Chapman2018}
\printbibliography[title = Referências]
\end{frame}
\begin{frame}
\frametitle{Sumário}
\tableofcontents
\end{frame}
\section{Introdução}
\begin{frame}
\frametitle{Introdução}
\begin{itemize}
  \item O computador foi (provavelmente) a invenção mais importante do século XX, pois afeta nossas vidas profundamente.
  \item Quando vamos ao supermercado, os scanners que verificam nossas compras são executados por computadores. 
  \item Nossos saldos bancários são mantidos por computadores, e os caixas automáticos e os cartões de crédito e débito que nos permitem fazer transações bancárias a qualquer hora do dia ou da noite são operados por mais computadores. 
  \item Os computadores controlam nossos sistemas telefônicos e de energia elétrica, acionam nossos fornos de micro-ondas e outros aparelhos e controlam os motores de nossos carros. 
  \item Quase todos os negócios no mundo desenvolvido entrariam em colapso da noite para o dia se fossem subitamente privados de seus computadores. 
  \item Considerando sua importância em nossas vidas, é quase impossível acreditar que os primeiros computadores eletrônicos foram inventados há apenas 75 anos.
  \end{itemize}
\end{frame}
\begin{frame}
\frametitle{Introdução}
  \begin{itemize}
  \item O que é esse dispositivo que teve tanto impacto em todas as nossas vidas? 
  \item Um computador é um tipo especial de máquina que armazena informações e pode realizar cálculos matemáticos sobre essas informações em velocidades muito mais rápidas do que os seres humanos podem pensar. 
  \item Um programa, que é armazenado na memória do computador, informa ao computador qual sequência de cálculos é necessária e em quais informações realizar os cálculos. 
  \item A maioria dos computadores são muito flexíveis. Por exemplo, o computador no qual escrevo essas palavras também pode equilibrar meu talão de cheques, se eu simplesmente executar um programa diferente nele.
  \end{itemize}
\end{frame}
\begin{frame}
\frametitle{Introdução}
  \begin{itemize}
  \item Os computadores podem armazenar grandes quantidades de informações e, com programação adequada, podem disponibilizar essas informações instantaneamente quando necessário. Por exemplo, o computador de um banco pode conter a lista completa de todos os depósitos e débitos feitos por cada um de seus clientes. 
  \item Em uma escala maior, as empresas de crédito usam seus computadores para armazenar os históricos de crédito de cada pessoa nos Estados Unidos – literalmente bilhões de informações. Quando solicitados, eles podem pesquisar esses bilhões de informações para recuperar os registros de crédito de qualquer pessoa e apresentar esses registros ao usuário em questão de segundos.
  \end{itemize}
\end{frame}
\begin{frame}
  \begin{itemize}
  \item É importante perceber que os computadores não pensam como os humanos entendem o pensamento. Eles apenas seguem os passos contidos em seus programas. 
  \item Quando um computador parece estar fazendo algo inteligente, é porque uma pessoa inteligente escreveu o programa que está executando. É aí que nós, humanos, entramos em ação. É nossa criatividade coletiva que permite que o computador realize seus aparentes milagres. 
  \item Este livro o ajudará a ensinar como escrever seus próprios programas, para que o computador faça o que você quer que ele faça.
\end{itemize}
\end{frame}
\section{O computador}
\begin{frame}
Um diagrama de blocos de um computador típico é mostrado na Figura 1-1. Os principais componentes do computador são a \textbf{unidade central de processamento}, \textbf{memória principal}, \textbf{memória secundária} e \textbf{dispositivos de entrada e saída}. 
\end{frame}
\section{Representação de dados em um Computador}
\begin{frame}
\end{frame}
\section{Linguagem de Computadores}
\begin{frame}
\end{frame}
\section{A história da linguagem fortran}
\begin{frame}
\end{frame}
\section{A evolução do fortran}
\begin{frame}
\end{frame}
\section{Resumo}
\begin{frame}
\end{frame}
\section{Execícios}
\begin{frame}
\end{frame}