\documentclass[aspectratio=169]{beamer}
\usepackage[T1]{fontenc}
\usepackage[utf8]{inputenc}
\usepackage{csquotes}
\usepackage[english,brazilian]{babel}
\usepackage[%
     style=abnt-numeric,%
     justify,%
]{biblatex}
\addbibresource{fortran_references.bib}
\begin{document}
\input{_0_apresentacao}
\input{_1_introducao}
\section{Elementos básicos de fortran}
\section{Projeto de Programa e Estruturas de Ramificação}
\section{Loops e manipulação de caracteres}
\section{Conceitos Básicos de Input/Output (I/O)}
\section{Introdução a matrizes (arrays)}
\section{Introdução aos Procedimentos}
\section{Recursos adicionais de Matrizes (arrays)}
\section{Recursos adicionais de Procedimentos}
\section{Mais sobre Variáveis do tipo Caracter}
\section{Tipos de dados intrínsecos adicionais}
\section{Tipos de dados derivados}
\section{Recursos avançados de procedimentos e módulos}
\section{Conceitos avançados de Input/Output}
\section{Ponteiros e estrutura de dados dinâmicas}
\section{Programação Orientada a Objetos em Fortran}
\section{Coarrays e processamento paralelo}
\section{Recursos de Fortran redundantes, obsoletos e excluídos}

\end{document} 