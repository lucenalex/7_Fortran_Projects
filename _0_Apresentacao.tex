\title{Linguagem de Programação Fortran}
\begin{frame}
\maketitle
\end{frame}
\begin{frame}
\begin{itemize}
  \item Ensinar os fundamentos da linguagem Fortran resultando em programas bons e de fácil manutenção.

  \item Passar por um processo de projeto detalhado antes que qualquer código seja escrito, usando uma técnica de projeto de cima para baixo para dividir o programa em partes lógicas que podem ser implementadas separadamente. 
  \item Enfatizar o uso de procedimentos para implementar essas partes individuais e a importância do teste de unidade antes que os procedimentos sejam combinados em um produto acabado. 
  \item Enfatizar a importância de testar exaustivamente o programa finalizado com muitos conjuntos de dados de entrada diferentes antes de ser liberado para uso.

  \item Ensinar o Fortran como é realmente encontrado por engenheiros e cientistas que trabalham na indústria e em laboratórios.

  \item Um fato da vida é comum em todos os ambientes de programação: grandes quantidades de código legado antigo que precisam ser mantidos.

\end{itemize}
\end{frame}
\begin{frame}
\begin{itemize}
  \item Esta edição baseia-se diretamente no sucesso do Fortran 95/2003 para Cientistas e Engenheiros, 3/e. Ele preserva a estrutura da edição anterior enquanto tece o novo material Fortran 2008 (e material limitado do padrão Fortran 2015 proposto) ao longo do texto. É incrível, mas o Fortran começou por volta de 1954 e ainda está evoluindo. A maioria das adições no Fortran 2008 são extensões lógicas dos recursos existentes do Fortran 2003 e estão integradas ao texto nos capítulos apropriados. No entanto, o uso de processamento paralelo e Coarray Fortran é completamente novo, e o Capítulo 17 foi adicionado para cobrir esse material. A grande maioria dos cursos de Fortran é limitada a um trimestre ou um semestre, e espera-se que o aluno aprenda tanto o básico da linguagem Fortran quanto o conceito de como programar. Esse curso cobriria os Capítulos 1 a 7 deste texto, além de tópicos selecionados nos Capítulos 8 e 9, se houver tempo. Isso fornece uma boa base para os alunos construírem em seu próprio tempo, à medida que usam o idioma em projetos práticos. Estudantes avançados e cientistas e engenheiros praticantes precisarão do material sobre números COMPLEXOS, tipos de dados derivados e ponteiros encontrados nos Capítulos 11 a 15. Cientistas e engenheiros praticantes quase certamente precisarão do material sobre recursos Fortran obsoletos, redundantes e excluídos encontrados no Capítulo 18. Esses materiais raramente são ensinados em sala de aula, mas são incluídos aqui para tornar o livro um texto de referência útil quando a linguagem é realmente usada para resolver problemas do mundo real.
\end{itemize}
\end{frame}