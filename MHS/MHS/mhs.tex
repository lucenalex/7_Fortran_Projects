\documentclass{article}
\usepackage{pgfplots}
\pgfplotsset{compat=1.17}
\usepgfplotslibrary{groupplots}

\begin{document}
	
	\begin{figure}
		\centering
		\begin{tikzpicture}
			\begin{axis}[
				xlabel=Tempo,
				ylabel=Posição,
				title=Gráfico do Oscilador Harmônico,
				grid=both,
				width=0.8\textwidth,
				height=0.6\textwidth
				]
				\addplot[smooth, very thick, red] table {oscilador.dat}; % Aqui ele importa os dados diretamente do arquivo oscilador.dat
			\end{axis}
		\end{tikzpicture}
		\caption{Gráfico do Oscilador Harmônico.}
	\end{figure}
	
	\begin{figure}
	\centering
	\begin{tikzpicture}
		\begin{axis}[
			xlabel=Tempo,
			ylabel=Posição,
			title=Gráfico do Oscilador Harmônico para Diferentes Frequências Angulares,
			grid=both,
			width=0.8\textwidth,
			height=0.6\textwidth,
			legend pos=north east, % Posição da legenda
			]
			\addlegendentry{$\omega = 1.0$}
			\addplot [smooth, very thick, blue] table {dados_freq_1.dat};
			
			\addlegendentry{$\omega =  2.0$}
			\addplot [smooth, very thick, red, dashed] table {dados_freq_2.dat};
			
			\addlegendentry{$\omega = 3.0$}
			\addplot [smooth, very thick, green, dotted] table {dados_freq_3.dat};
			
			\addlegendentry{$\omega = 4.0$}
			\addplot [smooth, very thick, orange, dashdotted] table {dados_freq_4.dat};
			
			\addlegendentry{$\omega = 5.0$}
			\addplot [smooth, very thick, purple, densely dotted] table {dados_freq_5.dat};
		\end{axis}
	\end{tikzpicture}
	\caption{Gráfico do Oscilador Harmônico para Diferentes Frequências Angulares.}
\end{figure}

\begin{figure}
	\centering
	\begin{tikzpicture}
		\begin{axis}[
			xlabel=Tempo,
			ylabel=Posição,
			title=Gráfico do Oscilador Harmônico para Diferentes Frequências Angulares,
			grid=both,
			width=0.8\textwidth,
			height=0.6\textwidth,
			legend style={
				at={(0.99,0.99)}, % Posição da legenda (ajustado para o canto superior direito)
				anchor = north east, % Ancoragem no canto inferior direito
				legend cell align = left, % Alinhamento das entradas da legenda
				legend columns = 1, % Número de colunas da legenda (ajustado de acordo com a quantidade de entradas)
				inner sep = 3pt,
				xshift=0pt, % Deslocamento horizontal negativo
				yshift=0pt, % Deslocamento vertical positivo
			},
			]
			\addlegendentry{$\omega = 1.0$}
			\addplot [smooth, very thick, blue] table {dados_freq_1.dat};
			
			\addlegendentry{$\omega = 2.0$}
			\addplot [smooth, very thick, red, dashed] table {dados_freq_2.dat};
			
			\addlegendentry{$\omega = 3.0$}
			\addplot [smooth, very thick, green, dotted] table {dados_freq_3.dat};
			
			\addlegendentry{$\omega = 4.0$}
			\addplot [smooth, very thick, orange, dashdotted] table {dados_freq_4.dat};
			
			\addlegendentry{$\omega = 5.0$}
			\addplot [smooth, very thick, purple, densely dotted] table {dados_freq_5.dat};
		\end{axis}
	\end{tikzpicture}
	\caption{Gráfico do Oscilador Harmônico para Diferentes Frequências Angulares.}
\end{figure}


\begin{figure}
	\centering
	\begin{tikzpicture}
		\begin{axis}[
			xlabel=Tempo,
			ylabel=Posição,
			title=Gráfico do Oscilador Harmônico para Diferentes Frequências Angulares,
			grid=both,
			width=0.8\textwidth,
			height=0.6\textwidth,
			legend style={at={(1.05,1)},anchor=north west}, % Posição da legenda fora do gráfico
			legend cell align=left, % Alinhamento das entradas da legenda
			]
			\addlegendentry{$\omega = 1.0$}
			\addplot [smooth, very thick, blue] table {dados_freq_1.dat};
			
			\addlegendentry{$\omega = 2.0$}
			\addplot [smooth, very thick, red, dashed] table {dados_freq_2.dat};
			
			\addlegendentry{$\omega = 3.0$}
			\addplot [smooth, very thick, green, dotted] table {dados_freq_3.dat};
			
			\addlegendentry{$\omega = 4.0$}
			\addplot [smooth, very thick, orange, dashdotted] table {dados_freq_4.dat};
			
			\addlegendentry{$\omega = 5.0$}
			\addplot [smooth, very thick, purple, densely dotted] table {dados_freq_5.dat};
		\end{axis}
	\end{tikzpicture}
	\caption{Gráfico do Oscilador Harmônico para Diferentes Frequências Angulares.}
\end{figure}

\begin{figure}
	\centering
	\begin{tikzpicture}
		\begin{axis}[
			xlabel=Tempo,
			ylabel=Posição,
			title=Gráfico do Oscilador Harmônico para Diferentes Frequências Angulares,
			grid=both,
			width=0.8\textwidth,
			height=0.6\textwidth,
			legend style={
				legend columns=-1, % Para formato de linha
				legend to name=legendname, % Nome da legenda
				at={(0.5,1.05)}, % Posição acima do gráfico
				anchor=south,
				/tikz/every even column/.append style={column sep=0.5cm} % Espaçamento entre as entradas da legenda
			},
			]
			\addlegendentry{$\omega = 1.0$}
			\addplot [smooth, very thick, blue] table {dados_freq_1.dat};
			
			\addlegendentry{$\omega = 2.0$}
			\addplot [smooth, very thick, red, dashed] table {dados_freq_2.dat};
			
			\addlegendentry{$\omega = 3.0$}
			\addplot [smooth, very thick, green, dotted] table {dados_freq_3.dat};
			
			\addlegendentry{$\omega = 4.0$}
			\addplot [smooth, very thick, orange, dashdotted] table {dados_freq_4.dat};
			
			\addlegendentry{$\omega = 5.0$}
			\addplot [smooth, very thick, purple, densely dotted] table {dados_freq_5.dat};
		\end{axis}
	\end{tikzpicture}
	\caption{Gráfico do Oscilador Harmônico para Diferentes Frequências Angulares.}
	\ref{legendname} % Referência para a legenda
\end{figure}

\begin{figure}
	\centering
	\begin{tikzpicture}
		\begin{axis}[
			xlabel=Tempo,
			ylabel=Posição,
			title=Gráfico do Oscilador Harmônico para Diferentes Frequências Angulares,
			grid=both,
			width=0.8\textwidth,
			height=0.6\textwidth,
			legend style={
				legend columns=-1, % Para formato de linha
				legend to name=legendname, % Nome da legenda
				at={(0.5,-0.2)}, % Posição abaixo do gráfico
				anchor=north,
				/tikz/every even column/.append style={column sep=0.5cm} % Espaçamento entre as entradas da legenda
			},
			]
			\addlegendentry{$\omega = 1.0$}
			\addplot [smooth, very thick, blue] table {dados_freq_1.dat};
			
			\addlegendentry{$\omega = 2.0$}
			\addplot [smooth, very thick, red, dashed] table {dados_freq_2.dat};
			
			\addlegendentry{$\omega = 3.0$}
			\addplot [smooth, very thick, green, dotted] table {dados_freq_3.dat};
			
			\addlegendentry{$\omega = 4.0$}
			\addplot [smooth, very thick, orange, dashdotted] table {dados_freq_4.dat};
			
			\addlegendentry{$\omega = 5.0$}
			\addplot [smooth, very thick, purple, densely dotted] table {dados_freq_5.dat};
		\end{axis}
	\end{tikzpicture}
	\ref{legendname}, % Referência para a legenda
	\caption{Gráfico do Oscilador Harmônico para Diferentes Frequências Angulares.}
\end{figure}

\begin{figure}
	\centering
	\begin{tikzpicture}
		\begin{axis}[
			xlabel=Tempo,
			ylabel=Posição,
			title=Gráfico do Oscilador Harmônico para Diferentes Frequências Angulares,
			grid=both,
			width=0.8\textwidth,
			height=0.6\textwidth,
			legend style={
				legend columns=5, % Para formato de linha
				at={(0.5,1.03)}, % Posição acima do gráfico, entre o frame e o título
				anchor=south,
				/tikz/every even column/.append style={column sep=0.1cm} % Espaçamento entre as entradas da legenda
			}
			]
			\addlegendentry{$\omega = 1.0$}
			\addplot [smooth, very thick, blue] table {dados_freq_1.dat};
			
			\addlegendentry{$\omega = 2.0$}
			\addplot [smooth, very thick, red, dashed] table {dados_freq_2.dat};
			
			\addlegendentry{$\omega = 3.0$}
			\addplot [smooth, very thick, green, dotted] table {dados_freq_3.dat};
			
			\addlegendentry{$\omega = 4.0$}
			\addplot [smooth, very thick, orange, dashdotted] table {dados_freq_4.dat};
			
			\addlegendentry{$\omega = 5.0$}
			\addplot [smooth, very thick, purple, densely dotted] table {dados_freq_5.dat};
		\end{axis}
	\end{tikzpicture}
	\caption{Gráfico do Oscilador Harmônico para Diferentes Frequências Angulares.}
	
\end{figure}


	
\end{document}
